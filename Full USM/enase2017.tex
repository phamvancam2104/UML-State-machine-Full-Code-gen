\documentclass[a4paper,twoside]{article}

\usepackage{epsfig}
\usepackage{subfigure}
\usepackage{calc}
\usepackage{amssymb}
\usepackage{amstext}
\usepackage{amsmath}
\usepackage{amsthm}
\usepackage{multicol}
\usepackage{pslatex}
\usepackage{apalike}
\usepackage{SCITEPRESS}     % Please add other packages that you may need BEFORE the SCITEPRESS.sty package.




\subfigtopskip=0pt
\subfigcapskip=0pt
\subfigbottomskip=0pt
\usepackage{filecontents}
\usepackage{listings}
\subfigtopskip=0pt
\subfigcapskip=0pt
\subfigbottomskip=0pt
\usepackage{lipsum}
\newcommand{\ti}{\textit}
\newcommand{\tb}{\textbf}
\newcommand{\tf}{\textsf}
\newcommand{\ttt}{\textit}

\usepackage{balance}

\usepackage{multirow}
\usepackage{multicol}
\usepackage{lipsum}
\usepackage{color}
\usepackage{xcolor}
\usepackage{colortbl}
\definecolor{LightCyan}{rgb}{0.88,1,1}
\definecolor{BluishGray}{RGB}{226,227,231}
\definecolor{LightGray}{gray}{0.95}
\definecolor{Gray}{gray}{0.85}
\usepackage{algorithm,xpatch}
\usepackage[noend]{algpseudocode}
\usepackage{comment}
\newcommand*\DNA{\textsc{dna}}
\usepackage{url}


\usepackage[hyphenbreaks]{breakurl}






\usepackage{amsthm}
\theoremstyle{definition}
\newtheorem{definition}{Definition}[section]

\usepackage{titlesec}
\titlespacing{\section}{0pt}{1.5ex plus 1ex minus .2ex}{1.5ex plus .2ex}
\titlespacing{\subsection}{0pt}{1.5ex plus 0.2ex minus .2ex}{1.5ex plus 0.2ex minus .2ex}
\titlespacing{\subsubsection}{20pt}{0.2\parskip}{-0.2\parskip}
%\titlespacing{\paragraph}{20pt}{0.2\parskip}{-0.2\parskip}

%to change to spacing between floats (figures) and text
%\setlength{\textfloatsep}{5pt plus 1.0pt minus 2.0pt}

\xpatchcmd{\algorithmic}{\setcounter}{\algorithmicfont\setcounter}{}{}
\providecommand{\algorithmicfont}{}
\providecommand{\setalgorithmicfont}[1]{\renewcommand{\algorithmicfont}{#1}}
\renewcommand{\algorithmiccomment}[1]{{\tiny\hfill$\triangleright$ #1}}
\setalgorithmicfont{\small}

\makeatletter
\def\BState{\State\hskip-\ALG@thistlm}
\makeatother



\usepackage{mdframed}

\definecolor{mygreen}{rgb}{0,0.6,0}
\definecolor{mygray}{rgb}{0.5,0.5,0.5}
\definecolor{mymauve}{rgb}{0.58,0,0.82}
\lstset{ %
	backgroundcolor=\color{white},   % choose the background color; you must add \usepackage{color} or \usepackage{xcolor}
	basicstyle=\scriptsize,        % the size of the fonts that are used for the code
	lineskip={-5.5pt},
	breakatwhitespace=false,         % sets if automatic breaks should only happen at whitespace
	breaklines=true,                 % sets automatic line breaking
	captionpos=t,                    % sets the caption-position to bottom
	commentstyle=\color{mygreen},    % comment style
	deletekeywords={...},            % if you want to delete keywords from the given language
	escapeinside={\%*}{*)},          % if you want to add LaTeX within your code
	extendedchars=true,              % lets you use non-ASCII characters; for 8-bits encodings only, does not work with UTF-8
	frame=single,	                   % adds a frame around the code
	keepspaces=true,                 % keeps spaces in text, useful for keeping indentation of code (possibly needs columns=flexible)
	keywordstyle=\textbf,       % keyword style
	language=Octave,                 % the language of the code
	otherkeywords={*, shallow,history,final,transition,table,initial,machine,transition,choice,exit,point,entry...},           % if you want to add more keywords to the set
	numbers=left,                    % where to put the line-numbers; possible values are (none, left, right)
	numbersep=5pt,                   % how far the line-numbers are from the code
	numberstyle=\tiny\color{mygray}, % the style that is used for the line-numbers
	rulecolor=\color{black},         % if not set, the frame-color may be changed on line-breaks within not-black text (e.g. comments (green here))
	showspaces=false,                % show spaces everywhere adding particular underscores; it overrides 'showstringspaces'
	showstringspaces=false,          % underline spaces within strings only
	showtabs=false,                  % show tabs within strings adding particular underscores
	stepnumber=2,                    % the step between two line-numbers. If it's 1, each line will be numbered
	frame=none,
	xleftmargin=3.0ex,
	stringstyle=\color{mymauve},     % string literal style
	tabsize=2,	                   % sets default tabsize to 2 spaces
	title=\lstname                   % show the filename of files included with \lstinputlisting; also try caption instead of title
}


\begin{document}
	
	\title{Complete Code Generation from UML State Machine }
	
	\author{\authorname{Van Cam Pham, Ansgar Radermacher, S\'ebastien G\'erard, Shuai Li}
		\affiliation{CEA LIST, Saclay, France}
		\email{\{first-name\}.\{last-name\}cea.fr}
	}
	
	\keywords{UML State Machine, code generation, semantics-conformance, efficiency, events, C++}
	
	\abstract{An event-driven architecture is a useful way to design and implement complex systems. The UML State Machine and its visualizations are a powerful means to the modeling of the logical behavior of such an architecture. In Model Driven Engineering, executable code can be automatically generated from state machines. However, existing generation approaches and tools from UML State Machines are still limited to simple cases, especially when considering concurrency and pseudo states such as history, junction, and event types. 
	This paper provides a pattern and tool for complete and efficient code generation approach from UML State Machine. It extends IF-ELSE-SWITCH constructions of programming languages with concurrency support. 
	The code generated with our approach has been executed with a set of state-machine examples that are part of a test-suite described in the recent OMG standard Precise Semantics Of State Machine.
	The traced execution results comply with the standard and are a good hint that the execution is semantically correct. 
	The generated code is also efficient: it supports multi-thread-based concurrency, and the (static and dynamic) efficiency of generated code is improved compared to considered approaches.}
	
	\onecolumn \maketitle \normalsize \vfill
	
	
	\section{\uppercase{Introduction}}
\label{sec:intro}
%1. Say the complexity, CPS .. UML SM is ...
%2. MDE ..., OMG defines powerful specification for USM..
%3. To blur the boundaries between... => need to have code generation
%4. Problem: 1. most of existing... focus only on sequency aspect..., many pseudo states are not supported while these pseudo states are widely used as powerful means to describe the dynamic behavior of system. 2. Concurrency is not taken into account. 3. Generated code is dependent on library of generation tools => not portable. Speed + executable size + runtime execution memory consumptio (for IoT)n + semantic conformance.
%To.... => this paper address the analysis for multithread-based concurrency code genration and the code generation for full UML state machine with respect to the specification...
%Contribution: Concurrency, systematic approach for code generation, evaluation conforming to UML precise semantics...
%Structure

%Internet of Things \cite{Li2015} drives the complexity of embedded systems today rapidly increases. 
%Today embedded systems directly interacting with running environment does not run in standalone mode but also react to the system environment changes. 
%Event-driven architecture \cite{Michelson2006} is an useful way to design such systems, in which events come to the system either from outside (data) or inside the system itself (time event or internal changes). 
The UML State Machine (USM) \cite{Specification2007} and its visualizations are efficient to model the behavior of event-driven architecture.   
%USM defines how systems behave in case of changes. 
%The rise in use of the Model-Driven Engineering (MDE) approach promotes the automation in software development. 
Tools and approaches are proposed to automatically translate USMs into executable code in the context of Model-Driven Engineering (MDE) \cite{Mussbacher2014}.%, which are recognized as very efficient for dealing with system complexity. 
%Automation is to reduce the gap between the modeling world
%, which consists of software architects, who prefer using graphical languages such as USM, 
%and the implementation world
%, which involves programmers 
%by bringing the ability to automatically generating code from diagram-based modeling languages such as USM to executable code.  
%The gap between the modeling world, which consists of software architects, who prefer using graphical languages such as USM, and the implementation world, which involves programmers, is therefore reduced. 

However, despite many advantages of MDE and USM, they are not widely adopted as a recent survey revealed \cite{1030}.
This is partially due to poor support for code generation \cite{forward2010perceptions}.

On one hand, the usefulness and semantics of USM are being empowered by OMG by providing more concepts and their precise semantics such as pseudo states and composite state machines. 
On the other hand, existing code generation tools and approaches have some issues regarding completeness, semantics and efficiency of generated code. 
Existing approaches either support a subset of USM modeling concepts or handle composite state machines by flattening into simple ones with a combinatorial explosion of states, and excessive generated code \cite{badreddin2014enhanced}.
%is language-dependent and still limited to simple cases, especially when considering concurrency of \ti{doActivity} and orthogonal regions, pseudo states such as history, and different events. 
%This again enlarges the gap between the USM semantics and the actual generated code.
Specifically, the following lists some of the current issues: 
\vskip 0.1cm
\noindent
\tb{Completeness:} Existing tools and approaches mainly focus on the sequential aspect while the concurrency of state machines is limitedly supported. 
Pseudo states are not rigorously supported by existing tools such as Rhapsody 
	%does not support \ti{doActivity}, junctions, and truly concurrent execution of orthogonal regions
	\cite{ibmdiff}.
	%Concurrency of is often implemented sequentially. 
	%Sinelabore \cite{sinelabore} does not support fork, join, junction and concurrent states. 
	%Rhapsody and Enterprise Architect \cite{EA} only support \ti{CallEvent} and \ti{TimeEvent} while \ti{SignalEvent} and \ti{ChangeEvent} are missing. 
	%Pseudo states such as history, choice and junction are poor \cite{EA,sinelabore} while these are very helpful in modeling.
Designers are then restricted to a subset of USM concepts during design.

\vskip 0.1cm
\noindent	
\tb{Efficiency:} Code generated from tools such as Rhapsody \cite{ibm_rhapsody} and FXU \cite{Pilitowski2007} depends on the libraries provided by the tool vendor, which makes the generated code non portable. 
Event processing speed and executable file size of generated code are not optimized \cite{6195875}.
	
\vskip 0.1cm
\noindent
\tb{Semantics:} The semantics of UML State Machine is defined by a recent OMG-standardized: Precise Semantics of State Machine (PSSM) \cite{OMG2015}.
This standard is not (yet) taken into account for validating the runtime execution semantics of generated code. 

%In order for wider integrating MDE into software development today, one task is to seamlessly make the behavior of the code generated from UML State Machine complied with the semantics specified by PSSM. 
Given the above issues, the objective of this paper is to present a novel code generation pattern and its tooling support. 
The latter offers efficient code generated from USMs with full concepts to reduce the modeling-implementation gap.   

The proposed pattern extends IF-ELSE constructions with our support for concurrency. 
%The tool also supports \ti{deep separation} for separating USM-prescribed code and user action code,
Runtime execution of generated code is experimented with the PSSM test suite.
%Although supporting multi-thread-based concurrency, binary files compiled from and the event processing speed of runtime execution of code generated by our approach are dramatically smaller than some manual implementation approaches and code generation tools.
  
%Abstraction provides simplified and focused views of a system and requires adequate graphical modeling languages such as Uni. Even, if the latter is not the silver bullet for all software related concerns, it provides better support than text-based solutions for some concerns such as architecture and logical behavior of application development. UML state machines (USMs) and their visual representations are much more suitable to describe logical behaviors of system entities than any equivalent text based descriptions. The gap from USMs to system implementation is reduced by the ability of automatically generating code from USMs  \cite{Booch1998, Douglass1999,Shalyto2006,Douglass1999}. 

%Ideally, a full model-centric approach is preferred by MDE community due to its advantages \cite{Selic2012}. However, in industrial practice, there is significant reticence \cite{Hutchinson:2011:MEP:1985793.1985882} to adopt it.
%On one hand, programmers prefer to
%use the more familiar textual programming language. 
%On the other hand, software architects, working at higher levels
%of abstraction, tend to favor the use of models, and therefore
%prefer graphical languages for describing the architecture of
%the system.
%However, on the one hand, maintaining code generated from existing approaches is non-trivial. On the other hand our observation is that it is very difficult to come up with formalizations that yield such elegant code generation solutions \cite{6032552}. In other words, generated code must be manually modified to build fully operational applications. 
%On one hand there are traditional developers who prefer to implement the system by writing code, while on the other hand there are developers who prefer to use entirely models for the design and implementation of the system. 
%The code modified by programmers and the model are then inconsistent. Round-trip engineering (RTE) \cite{Hettel2008} is proposed to synchronize different software artifacts, model and code in this case \cite{Sendall}. RTE enables actors (software architect and programmers) to freely move between different representations \cite{Sendall} and stay efficient with their favorite working environment. 

%After code modifications, round-trip engineering (RTE) is needed to make the model and code consistent, which is a critical aspect to meet quality and performance constraint required from project managers today. 
%Unfortunately, current industrial tools such as for instance Enterprise Architect \cite{sparxsystems_enterprise_2014} and IBM Rhapsody\cite{ibm_rhapsody} only support structural concepts for RTE such as those available from class diagrams and code. Compared to RTE of class diagrams and code, RTE of USMs and code is non-trivial. It requires a semantical analysis of the source code, code pattern detection and mapping patterns into USM elements. 
%This is a hard task, since mainstream programming languages such as C++ and JAVA do not have a trivial mapping between USM elements and source code statements.

%For software development, one may wonder whether this RTE is doable. That is, why do the industrial tools not support the propagation of source code modifications back to original state machines? Several possible reasons to this lack are (1) the gap between USMs and code, (2) not every source code modification can be reverse engineered back to the original model, and (3) the penalty of using transformation patterns facilitating the reverse engineering that may not be the most efficient (e.g. a slightly larger memory overhead). 
%in the mind of these tools' vendors, users always make changes to models rather than to code. Generated code, in these tools, is therefore not supposed to be changed directly.  

%In this paper, we address the RTE of UML State Machine diagrams and its related generated code. We propose a RTE approach consisting of a forward process which generates code by using transformation patterns, and a backward process which is based on code pattern detection to update the original state machine model from the modified generated code. From the proposed approach, we implemented a prototype and conducted several experiments on different aspects of the round-trip engineering to verify the proposed approach. 



%Model-driven engineering (MDE) is a development methodology aiming to increase software productivity and quality by allowing different stakeholders to contribute to the system description \cite{Mussbacher2014}. MDE considers models as first-class artifacts and generates code from higher abstraction level models. Recent survey \cite{1030} has revealed that industries are gaining the adoption of code generation into software development life-cycle. Although many tools and research prototypes can generate executable code from models, generated code could be manually modified by programmers, e.g. skeleton code generated from UML \cite{Specification2007} class diagrams. Models and the generated code are therefore out of synchronization. Round-trip engineering \cite{Aßmann200333, Hettel2008, E-ESE-120044648} (RTE) is proposed to keep the artifacts synchronized.

%RTE supports synchronizing different software artifacts, model and code in this case, and thus enabling actors (software architect and programmers) to freely move between different representations \cite{Sendall}. Tools such as for instance Enterprise Architect \cite{sparxsystems_enterprise_2014}, Visual Paradigm \cite{visual}, and AndroMDA \cite{_andromda_} provide RTE but most of them are only applicable for system structure models such as class diagrams.  

%This study addresses the RTE of UML State Machine (SM) and object-oriented programming languages such as C++ and JAVA. SM is widely used in practice for modeling the behavior of complex systems, notably reactive, real-time embedded systems. There are several approaches to generating source code from state machines or state charts such as nested switch/if statements \cite{Booch1998}, state-event-table \cite{Douglass1999, Duby2001}, and state pattern \cite{Allegrini2002,Shalyto2006,Douglass1999}. Unfortunately, the generated code from these approaches is very difficult for programmers to maintain without an appropriate supporting tool. RTE is impossible in these approaches even with very small changes such as changing transition targets or actions made to code. The reason behind this impossibility is that, in mainstream programming languages such as C++, JAVA, (1) there are not equivalents between SMs and source code statements and (2) the code generation pattern of these approaches has not been chosen with RTE in mind.

%This paper addresses the RTE of UML state-machines and object-oriented programming languages such as C++ and JAVA. The forward  engineering of the approach takes as input a state-machine and executes two transformations. The first is UML to UML by utilizing several transformation patterns such as the double-dispatch approach presented in \cite{spinke_object-oriented_2013} and the second is a generation of code from the transformed UML. Traceability information is stored, during the transformations. In the backward direction, a verification is executed by the code pattern detection to verify the correctness of the code before the backward process taking as input the modified generated code, the UML classes, the original state-machine and mapping information together merges changes from code to the state-machine. We implemented a prototype supporting RTE of state-machine and C++ code, and conducted several experiments on different aspects of the RTE to verify the proposed approach. To the best of our knowledge, our implementation is the first tool supporting RTE of SM and code. 
%The prototype also improves the collaboration between MDE developers and traditional programmers in developing reactive complex embedded systems.

%This paper addresses the RTE of USMs and object-oriented programming languages such as C++ and JAVA. The main idea is to utilize transformation patterns from USMs to source code that aggregates code segments associated with a USM element into source code methods/classes rather than scatters these segments in different places. Therefore, the reverse direction of the RTE can easily statically analyze the generated code by using code pattern detection and maps the code segments back to USM elements. Specifically, in the forward direction, we extend the double dispatch pattern presented in \cite{spinke_object-oriented_2013}. Traceability information is stored during the transformations. We implemented a prototype supporting RTE of state-machine and C++ code, and conducted several experiments on different aspects of the RTE to verify the proposed approach. To1 the best of our knowledge, our implementation is the first tool supporting RTE of SM and code. 

To sum up, the contributions of this paper are: (1) an approach and tooling support for code generation from USMs with full features; 
(2) an empirical study on the semantic-conformance and efficiency of generated code;
and (3) application of the tool to a case study.  

We assume that readers of this paper have knowledge about UML State Machine and its basic execution semantics.

The remaining of this paper is organized as follows: Section \ref{sec:modeling} describes the modeling of applications using UML State Machines. 
Section \ref{sec:uniqueness} mentions the features of our tool.
Thread-based concurrency is designed in Section \ref{sec:thread}. 
Based on this design, a code generation approach is proposed in Section \ref{sec:codegen}. 
The implementation and empirical evaluation are reported in Section \ref{sec:exp}. 
The application of our tool to a case study is presented in Section \ref{sec:casestudy}.
Section \ref{sec:relatedwork} discusses related work. The conclusion and future work are presented in Section \ref{sec:conclusion}.
	
	\section{\uppercase{State Machines and UML events}}
\label{sec:modeling}
This section presents overview of using UML State Machines for modeling and designing reactive software applications. 
A state machine is used for describing the behavior of either a class in object-oriented design or a component in component-based design.
In the following, we commonly use the term \ttt{class}.

The state machine processes external and internal events.
UML defines four event types: \ttt{CallEvent, SignalEvent, TimeEvent, ChangeEvent}.
A call event is associated with an operation/method and emitted if the operation is invoked.
The processing of call events is synchronous meaning that it runs within the thread of the operation caller.
The processing of other events is asynchronous meaning that these events received by the class are stored in an event queue which is maintained by the class at runtime for later processing.
A signal event is associated with a UML signal type containing data.
It is emitted if the class receives an instance of the signal type.
From a programming perspective, we provide an API \ttt{sendSignal} to send the signal instance from environment code or other classes to the class and store the event in the queue.                                             
		
A time event specifies the time of occurrence relative to a starting time. 
The latter is defined as the time when a state with an outgoing transition triggered by the time event is entered.
The time event is emitted if this accepting state remains active longer that the relative time of occurrence. 
Once emitted, it triggers the transition.
In other words, the state, which is the source vertex of a transition triggered by a time event, will remain active for a maximal amount of time specified by the time event.	
A change event has a boolean expression and is fired if the expression's value changes from false to true. 
Note that unlike call and signal events, time and change events are automatically fired inside the class.

\vskip 0.1cm
\noindent
\tb{Deferred events}: A state can specify to defer some events.
It means that if an event specified as deferred, it will be not processed while the state remains active.
The deference of events is used to postpone the processing of some low-priority events while the state machine is in a certain state.
%The deferred event will be pushed back to the event queue if another event in the event queue is processed.


We support all of these events to model event-driven reactive applications.
                                        
	
	\section{\uppercase{Features}}
\label{sec:uniqueness}
Our pattern and tool has some features compared to other tools as followings:

\vskip 0.1cm
\noindent
\tb{Completeness:} Our tool supports all state machine vertexes and transitions including all pseudo states and transition kinds such as external, local, and internal. 
Hence, the tool improves flexibility of using UML State Machines to express architecture behavior.
For the moment, our tool cannot deal with transitions from an \ti{entry point} to an \ti{exit point}.
We believe that these transitions are not used in reality.
This is because the contradictory semantics of \ti{entry points} and \ti{exit points}. 
In UML, \ti{entry points} and \ti{exit points} represent entering points and exit points of a compoiste state, respectively. 
They provide encapsulation of the insides of the state. 
The \ti{entry points} allow users to customize the way to enter the composite state instead of the default entering way while the \ti{exit points} allow to customize the exiting way.
For example, the \ti{Enp} entry point in Fig. \ref{fig:entering} allows the \ti{S5} sub-state of the \ti{S1} composite state to be active instead of \ti{S3} by the default entering way.

\vskip 0.1cm
\noindent	
\tb{Event support:} Our tool promotes four UML event types and event deference mechanism, which are able to express synchronous and asynchronous behaviors and exchange data between components/classes.

\vskip 0.1cm
\noindent	
\tb{UML-conformance:} A recent specification formalizing the Precise Semantics of UML State Machine (PSSM) is under standardization of the OMG.
It defines a test suite with 66 test cases for validating the conformance of runtime execution of code generated from UML State Machines.
We have experimented our tool with the test suite.
Traced execution results of 62/66 test cases comply with the standard and are, therefore, a good hint that the execution is semantically correct.
%Due to space limitation, the details of patterns and evaluation for state machine code generation semantics are not presented in this paper.

\vskip 0.1cm
\noindent	
\tb{State machine configuration:} 
Asynchronous events such as signal events, change events, and time events are stored in an event queue.
%Change expressions of change events are monitored and periodically evaluated\footnote{Currently, this discrete evaluation mechanism is not recommended for critical systems since the monitor might miss change event occurrences between two evaluations.} to track their values.
A signal event can bring data (message).
Our tool allows to configure the event queue size and the maximal size of signals.
The configuration is not specified by UML because the specification wants to be abstract.
We allow to determine these values through a specific profile.
Note that the configuration information might not be needed in dynamic memory allocation.
The latter, however, is not recommended in embedded systems.
%Fig. \ref{fig:entering} shows the configuration stereotype annotated on the state machine example.

\vskip 0.1cm
\noindent
\tb{Efficiency:} We conducted experiments on some benchmarks to show that code generated by our tool is efficient and can be used to develop resource-constrained embedded software.
Specifically, event processing is fast and the size of executable files compiled from generated code is small.

\vskip 0.1cm
\noindent
\tb{Event API:}
Generated code in our tool provides APIs for environment code to invoke operations or send data signals to reactive classes.
The invocations and sending will automatically fire events for state machines to process.

\vskip 0.1cm
\noindent
\tb{Concurrency:} 
Concurrency aspects in state machines including doActivity of states, orthogonal regions, event detection, and event queue management are handled by the execution of multiple threads.
Currently, we use POSIX threads for concurrency.

\vskip 0.1cm
\noindent
\tb{Portability:}
Currently, our tool generates C++ code.
The generated code can run on POSIX systems such as Ubuntu without installing any additional libraries to be able to compile and execute the code.
Our code generation pattern and tool can be extended to generate code in other programming languages such as Java which supports threads and mutexes for multi-thread synchronization.
	%\input{sections/formalizerevised}
	
	\input{sections/concurrency}
	
	\section{Code generation}
%\subsection{Transformation pattern}
%todo: describe the pattern

%Transformation from State machine to fUML (classes, attributes, methods)
%\lipsum[1]

\subsection{Assumption}
%todo: give some assumptions on code generation such as functions to create methods, attriutes, classes
To give the formalization of the code generation, we assume that we want to generate from the state machine to an object oriented programming language $ActLang$. Assuming that our code generator contains primitive functions supporting for generating $ActLang$ as following:
\begin{itemize}
	\item $genClass(n, generals, itfs)$ creates a class with its name, parent class set, and implemented interfaces as \ti{n}, \ti{generals}, and \ti{itfs}.
	
	\item $genMtd(n, c, type, params)$ creates a method $m$ with its name as $n$ inside the class $c$, its return type as $type$, and $params$ as its parameter set.
	
	\item $genAttr(n, c, type, multiplicity)$ creates an attribute named $n$ in the class $c$ and typed by $type$. The create attribute is an array if $multiplicity > 1$, otherwise a simple attribute.
	
	\item $genEnum(n)$ and $genEnumLit(enum, n)$ create an enumeration and its enumeration literal, respectively.
	
	\item $genBody(m, body)$ adds a body to a method. The body is a string which contains a list of statements.
	
	\item $createParalle(t, seg)$ generates a mechanism which allows the segment code $seg$ run in a thread $t$. Similarly, $genWait(t), genJoin(t)$.
	
	\item $genMutex(size)$ creates an array of mutexes with \ti{size} as the number of items of the array. 
	
	\item $synchronize(seg)$ generates a mechanism which allows the segment code $seg$ run safely (can be either based on \ti{POSIX pthread} or \ti{Java synchronize} mechanism).
	
	\item $toString(stts)$ is used to convert a list of statements $stts$ into a readable string which can be add to a method as its body.
	
	\item Concatenation of two strings $str1$ and $str2$ is concisely described as $str1 + str2$.
	
	\item \ti{WHILE}, \ti{FOR} \ti{IF}, \ti{ELSE} are symbols representing while and for loops, if and else statements.
	
	\item \ti{FORK(func)} creates a thread (lightweight process) associated with the function/method \ti{func} and \ti{JOIN(theThread)} waits until the method associated with the thread \ti{theThread} completes.
\end{itemize} 

\subsection{Code generation algorithm}
\subsubsection{State transformation}
Suppose that we want to generate a state machine $sm$ whose states are listed by $lstates$. A common state interface $IState$ is created. The interface contains three methods, namely, \ti{entry}, \ti{exit}, and \ti{doActivity} corresponding to three state actions, respectively. To preserve the hierarchy of composite states, the interface also has two attributes called \ti{activeStates} and \ti{previousStates} referring to active sub-states \ti{actives} , previous active sub-states \ti{previousStates} in case of the presence of history states, and a list of deferred event identifiers.

Each UML state is transformed into an instance of the interface associated with a state ID (which is a child element of an enumeration) inside the active class $C$. When initialization, each instance refers its methods to the actual methods implemented in $C$. In C++, this referring is done by using the powerful mechanism function pointer. In other object-oriented languages such as Java, this is done with anonymous sub-classes of the interface. Listing \ref{lst:IStateCpp} and \ref{lst:IStateJava} show the interface and its instances associated with the states of the state machine in C++ and Java, respectively, in which S0 is one of $lstates$. \ti{NUM\_STATES} is the number of states in the state machine. The actions of the states are implemented in the active class $C$ and named depending on the name of the states. In the following sections, we only consider C++ as out \ti{ActLang}. The discussion of other object-oriented languages are much similar since these share the same concepts,  

\begin{lstlisting}[caption=IState interface and function pointers in C++, label=lst:IStateCpp, frame=single]
typedef struct IState {
  IState** previousStates; 
  IState** actives;
  EventId* defEvents;
  void (C::*entry)();
  void (C::*exit)();
  void (C::*doActivity)();
} IState;
class C {
private:
  IState states[NUM_STATES];
public:
  C() {
    states[S0_ID].entry = &C::S0_entry;
    ...
  }
  void S0_entry {...}
}
\end{lstlisting}

\begin{lstlisting}[mathescape=true, caption=IState interface and annonymous sub-classes in Java, label=lst:IStateJava, frame=single]
public interface IState {
  public IState[] previousStates; 
  public IState[] actives;
  public EventId defEvents;
  public void entry();
  public void exit();
  public void doActivity();
}
class C {
private IState states[NUM_STATES];
public C() {
  states[S0_ID] = new IState() {
    public void entry() {
      S0_entry();
    }
    ...
  }
}
public void S0_entry() {...}
}
\end{lstlisting}

The procedure to generate the code for states is shown in Listing \ref{lst:procedure1}. It first creates the state interface $IState$ (in C++, it is either a class or a struct). The array attribute is then created with the number of states as its size. Each state is also associated with a state ID which is a child of an enumeration. Finally, the constructor of $C$ is created to initialize and make methods of the attribute instances refer to \ti{entry/exit/doActivity} action methods of $C$. The implementation of action methods in the context class $C$ is similar to the delegation pattern proposed by the authors in \cite{Niaz2004} but dramatically decreases the memory consumption since only one common interface for all states is created instead of a class for each state in \cite{Niaz2004}.

\begin{lstlisting}[mathescape=true, caption=Procedure to create code for states, label=lst:procedure1, frame=single]
IState = genClass('IState', $\emptyset$, $\emptyset$);
stateIdEnum = genEnum('StateIdEnum');
foreach s in lstates
  genEnumLit(stateIdEnum, s.name + '_ID');
  mtd = genMtd(s.name + '_entry', C, 
			null, null);
  genBody(mtd, toString(entry(s)));
  ...
genEnumLit(stateIdEnum, 'NUM_STATES');  
genAttr('states', C, IState, NUM_STATES); 
genMtd(C.name, C, null, null);
\end{lstlisting}

\subsubsection{Region transformation}


\subsubsection{Event transformation}
An event enumeration \ti{EventId} is created whose children are event identifiers associated with events. Each event is also transformed into a method in the context class $C$. Suppose $levents$ is the list of events which can be processed by the state machine $sm$. Besides the explicitly defined events of the state machines, $levents$ contains a special event called $CompletionEvent$. The latter is, following the UML specification, an implicit event triggering triggerless transitions. It is emitted when either \ti{doActivity} of an atomic state finishes its execution or all orthogonal regions of a composite state have reached to a final state. 

UML defines five types of events including \ti{CallEvent}, \ti{SignalEvent}, \ti{TimeEvent}, \ti{ChangeEvent}, and \ti{Any}. A transition triggered by an \ti{Any} event is meant to be fired by any of the other events. To process events, for each event, a method is implemented in $C$. Each event triggers a list of transitions. We suppose $T_{trig}(e)$ is the transition list triggered by the event $e$, and $S_{trig}(e) = \{src(t) | t \in T_{trig}(e)\}$. In other words, $S_{trig(e)}$ is a set of states which are the source states of the transitions in $T_{trig}(e)$. To present how the body of event methods is generated, we define functions as followings:
\begin{itemize}
	\item Vertex depth $dp(v)$ is defined as:
			\begin{equation}
			dp(v) =    \left\{
			\begin{array}{ll}
			1 & \ti{v is a root vertex}  \\
			dp(ctner(v)) + 1& otherwise \\
			\end{array} 
			\right.
			\end{equation}
	\item $Map_{e}(s) \subset S_{trig(e)} | \forall sub \in map_e(s): ctner(sub) = s$, $Prt(e) = \{s \in V| map_{e}(v) \neq \emptyset\}$. $Prt(e)$ is an ordered list whose length is $len(Prt\{e\})$ and elements are accessed by indexes. The order of $Prt(e)$ is defined as:	$\forall i, j \leq len(Prt\{e\})$, 
	\\ if $i < j, dp(Prt(e).get(i)) \geq dp(Prt(e).get(j))$. 	
\end{itemize}


The procedure in Listing \ref{lst:eventproc} describes how to generate the body of the method associated with an event. It generates the code checking for active states respecting the UML semantics in which the innermost states process the incoming event first. To do this, it first looks in the source state list $S_{trig(e)}$ for the innermost states that accept the event triggering its outgoing transitions. If these found states are children of a concurrent state, $genStateCheck$ generates the checking codes run in parallel, which will be described later in \ref{subsubsec:thread}. Otherwise said, sequential code is generated.

\begin{lstlisting}[mathescape=true, caption=Procedure to create code event processing, label=lst:eventproc, frame=single]
for item in $Lm(e)$
  if ($item.kind = conc$)
    for s in Map_e(item)
      genStateCheck() 
  else 
    for s in Map_e(item)
      genStateCheckWithElse  	
\end{lstlisting}



\subsubsection{Thread-based Concurrency}
\label{subsubsec:thread}
\paragraph{Thread-based concurrency analysis} 

While concurrency is an important aspect defined by the UML State machine specification, especially hierarchical and concurrent state machines with \ti{doActivity}s for states, most of existing approaches do not take into account. This is non-trivial since concurrency is dynamic in UML state machine since the number of threads used for concurrency is non-deterministic.

For example, assuming that \ti{Idle} is the current active state of the ATM state machine in Fig. 
\ref{fig:example} and a \ti{verifyPIN} event is coming. 
The \ti{doActivity} behavior of \ti{Idle} \ti{doActivity(Idle)} (if has) is terminated, \ti{exit(Idle)} and the \ti{effect(t2)} (\ti{T2Effect}) are executed sequentially. 
These actions are run in a state machine main thread which reads incoming events from a "first in, first out" (FIFO) priority queue. 
Fig. \ref{fig:threading1} shows the activity diagram representing the concurrency of the state machine example when processing the \ti{verifyPIN} event, in which each activity partition represents a thread. 
The completion of \ti{effect(t2)} is followed by \ti{effect(t3)} and \ti{effect(t3)}, which are run concurrently since the transitions owning these effects outgo from a fork pseudo state. 
Two threads \ti{T3Run} and \ti{T4Run} associated with \ti{effect(t2)} and \ti{effect(t3)}, respectively, are created by \ti{FORK}.
The entry action \ti{entry(Verifying)} of \ti{Verifying} is executed following the termination of the two threads. 

After \ti{entry(Verifying)} completion, the UML specification says that \ti{doActivity(Verifying)}, \ti{entry(VerifyingCard)} and \ti{entry(VerifyingPIN)} should be concurrently executed, which is represented by a fork node, in which a \ti{Start} signal is sent to \ti{VerifyingDoRun} in order for commencing \ti{doActivity(Verifying)}. 
As the \ti{Verifying} state, the \ti{doActivity}s of the states \ti{VerifyingCard} and \ti{VerifyingPIN} are also concurrently started. 
Also, upon the completion of \ti{entry(VerifyingCard)} and \ti{entry(VerifyingPIN)}, the main thread completes the processing of the \ti{verifyPIN} event, reads next events from the queue or waits for next event occurrences.

\begin{figure}
	\centering
	\includegraphics[clip, trim=1.0cm 1.6cm 1.6cm 1cm, width=1.03\columnwidth]{figures/ThreadingExample.pdf}
	\caption{Concurrency of the ATM when receiving the \ti{verifyingPIN} event} 
	\label{fig:threading1}
\end{figure}

If no event is coming, and \ti{doActivity(VerifyingCard)} and \ti{doActivity(VerifyingPIN)} are long actions (e.g. forever loops inside), the state machine remains its active configuration and three concurrent actions including \ti{CheckForEvents}, \ti{doActivity(VerifyingCard)}, and \ti{doActivity(VerifyingPIN)} are permanently run.

It is worth noting that the termination time of \ti{doActivity(VerifyingCard)} and \ti{doActivity(VerifyingPIN)} is non-deterministic. 
However, whenever one of those completes, a completion event associated with the state corresponding to the completed \ti{doActivity} is generated and pushed to the event queue. 
For illustration, assuming that \ti{doActivity(VerifyingCard)} terminates before \ti{doActivity(VerifyingPIN)}. 
As the activity diagram in Fig. \ref{fig:threading2}, the Main thread checks the \ti{CompletionEvent} upon the completion of \ti{doActivity(VerifyingCard)}. \ti{exit(VerifyingCard)} and \ti{effect(t5)} are then executed sequentially. If \ti{cardValid} is computed as true as the result of \ti{doActivity(VerifyingCard)} and \ti{exit(VerifyingCard)}, the Main thread simply executes \ti{effect(t6)} and \ti{entry(CardValid)} before waiting for other events.

In contrast, Main sends \ti{Stop} signals to stop \ti{doActivity(VerifyingPIN)} and \ti{doActivity(Verifying)}, executes exit actions, effects and entry actions in an appropriate order (see Fig. \ref{fig:threading2}) and waits for other events.

So far, we see that the number of concurrent actions is not constant but changes timely. 
Each action can either deterministically or non-deterministically terminate. 
In this sense, deterministic actions (DAs) prevent the Main thread from going to the waiting-for-event point. 
In other words, pending events in the queue are only read and processed once all deterministic actions complete. Therefore, we re-define the run-to-completion paradigm of UML state machine as following:
 
\begin{definition}
	Run-to-completion means that, in the absence of exceptions or asynchronous destruction of the context	class object or the state machine execution, a pending Event occurrence is dispatched only after the completion of all deterministic actions commenced by the processing of the current event. 
	At this point, a stable state configuration has been reached
\end{definition}

In the example, some of DAs are as followings: \ti{effect(t2)}, \ti{effect(t3)}, \ti{effect(t4)}, \ti{entry(Verifying)}, \ti{entry(VerifyingCard)}, \ti{entry(VerifyingPIN)} and non-deterministic actions (NDAs) as followings: \ti{doActivity(Verifying)}, \ti{doActivity(VerifyingCard)} and \ti{doActivity(VerifyingPIN)}.


\begin{figure}
	\centering
	\includegraphics[clip, trim=1.5cm 1.6cm 1.6cm 1cm, width=1.03\columnwidth]{figures/ThreadingExample2.pdf}
	\caption{Concurrency of the ATM when \ti{doActivity} of \ti{VerifyingCard} completes before that of \ti{VerifyingPIN}}
	\label{fig:threading2}
\end{figure}

\paragraph{Thread-based design of generated code}
Each NDA is run in parallel with the main thread which reads and dispatch events from the event queue. 
Each is associated with a thread which is initialized at the state machine initialization moment. 
The number of threads associated with NDAs is therefore equal to that of the NDAs.
The design of threads is based on the thread pool pattern, which initializes all threads at once, and the paradigm "wait-execute-wait". 
In the latter, a thread \tb{waits} for a signal to \tb{execute} its associated method and goes back to the \tb{wait} point if it receives a stop signal or its associated method completes. 
An NDA is one of the followings:
\begin{itemize}
	\item \ti{doActivity} of each state if has. The number of \ti{doActivity} $n_{do} = \#\{s \in V|\exists doActivity(s)\}$
	
	\item Sleep function associated with a \ti{TimeEvent} which counts ticks and emits a \ti{TimeEvent} once completes: $n_{te} = \#\{e \in E|\ti{e is a time event}\}$.
	
	\item Change detect function associated with a \ti{ChangeEvent} which observes a variable or a boolean expression and pushes an event to the queue if changes happen: $n_{che} = \#\{e \in E|\ti{e is a change event}\}$.
\end{itemize} 

Therefore, the concurrency has the number of initial threads $n_{threads} = n_{do} + n_{te} + n_{che}$ plus a main thread which sends start and stop signals to these initial threads. 

Now we consider spontaneous threads which are created by \ti{FORK} to run DAs, joined until and destroyed once DAs complete. The followings describe different types of DAs:

\begin{itemize}
	\item Actions executed when entering/exiting an orthogonal region, which can be: execute a chain of transition effects contained by the region before entering a stable sub-state or exiting the region completely: $n_{region threads} = \#\{r \in \mathcal{R}|ctner(r).kind=concurrent\}$
	
	\item Effects of transitions outgoing from a $fork$ and those incomings to a $join$: \\
	$\mathcal{J} = \{v \in V|v.kind=join\}$ \\
	$\mathcal{F} = \{v \in V|v.kind=fork\}$ \\
	$$n_{FJ\_threads} = \sum_{v \in \mathcal{F}} {\#T_{outs}(v)} + \sum_{v \in \mathcal{J}} {\#T_{ins}(v)}$$.
\end{itemize}

The spontaneous threads follow a paradigm in which if a thread $parent$ creates a set of threads $children$, $parent$ must wait until $children$ complete their associate methods. These threads are created in one of the following cases:

\begin{itemize}
	\item Having multiple transitions outgoing from a \ti{fork}, for each transition effect, a thread is created by \ti{FORK}
	
	\item Entering a concurrent state $s$, after the execution of $entry(s)$, a thread is also created for each orthogonal region. 
	
	\item Exiting a concurrent state $s$, before the execution of $exit(s)$, a thread is also created for each region to exit the corresponding active sub-state. 
\end{itemize}

\paragraph{Example of generated code}

 


 


	
	%\section{Applying deep separation to code generation}	
	
	\section{experiments}
\lipsum[1]
		


	
	\section{\uppercase{Traffic Light Controller simulation}}
\label{sec:casestudy}
In order to assess the usability and practicality of using UML State Machines and events, we applied our tool to a simplified Traffic Light Controller (TLC) system as a case study, which is extracted from \cite{katz2005contemporary}.

%\subsection{Traffic Light Controller}
\label{subsec:tlc}
TLC controls an intersection of a busy highway and a little-used farm-way as in Fig. \ref{fig:casestudy}.
Detectors are placed along a farmroad to raise the signal \ti{C} as long as a vehicle is waiting to cross the highway. 
The highway lights remains green as long as no vehicle is detected on the farmroad. 
Otherwise, the highway lights should change from yellow to red, allowing the farmroad lights to become green. 
The farmroad lights stay green only as long as a vehicle is detected on the farmroad and never longer than a set interval to allow the traffic to flow along the highway. 
If no vehicle or timeout expired, the farmroad lights change from green to yellow to red, allowing the highway lights to return to green. 
Even if vehicles are waiting to cross the highway, the highway should remain green for a set interval.




\begin{figure}
	\centering
	\includegraphics[clip, trim=0.6cm 12.5cm 10.9cm 1.8cm, width=1.0\columnwidth]{figures/casestudy}
	\caption{Traffic Light Controller (left) and its class diagram (right).} 
	\label{fig:casestudy}
\end{figure}


The object-oriented class diagram follows the design in Yasmine \cite{trafficlight}, which is a C++11 state machine framework, and is shown in Fig. \ref{fig:casestudy} (right).
%To emulate the development situation and apply RAOES, a software architect and a programmer participated to the development.
%The class system design is similar to the object-oriented one presented in \cite{trafficlight}.
The behavior of each class is described by a state machine.
%However, the state machine describing the behavior of \ttt{Intersection} in our design is specified by utilizing the deference of events.
The state machines of \ttt{Intersection} and \ttt{TrafficLight} are shown in Fig. \ref{fig:casestudystatemachine} (left and right, respectively).
All of the states of \ttt{IntersectionStateMachine}, except \ttt{FarmwayOpen}, are composite.
The details of \ttt{SwitchingHighwayToFarmroad} and \ttt{SwitchingFarmroadToHighway} are actually shown on the yasmine site \cite{trafficlight}.

\begin{figure}
	\centering
	\includegraphics[clip, trim=0.2cm 13.6cm 11.4cm 0.2cm, width=1.0\columnwidth]{figures/casestudystatemachine}
	\caption{State machines for describing the behavior of Intersection (left) and TrafficLight (right)} 
	\label{fig:casestudystatemachine}
\end{figure}


%alternative design for the HighwayOpen composite state
The conditions for switching from the state \ttt{HighwayOpen} to \ttt{SwitchingHighwayToFarmroad} are: (1) a minimum time for the highway open is elapsed; and (2) the sensors emit a signal.

To show the usability and practicality of UML events, two alternative designs can be specified by using time events and change events. 
Fig. \ref{fig:highwayopenalternatives} (a) and (b) show the alternates, respectively.
The first design in \ref{fig:highwayopenalternatives} (a) uses a time event, which triggers the transition from \ttt{WaitingForHighwayMinimum} to \ttt{MinimumTimeElapsed}, and a signal event deferred by the \ttt{WaitingForHighwayMinimum} state. 
When \ttt{HighwayOpen} becomes active, its active sub-state remains \ttt{WaitingForHighwayMinimum} as long as the minimum time.
If a signal C is fired from the detector, a signal event \ttt{DetectorOn} is sent to the state machine.
The event is, however, not immediately processed but delayed by until the active sub-state becomes \ttt{MinimumTimeElapsed} in case the time event is fired.
The signal event is then processed to finish the execution of \ttt{HighwayOpen} and activate the farmway.

The other design utilizes a change event instead of deferred events for switching from \ttt{WaitForPreconditions} to a final state. 
Two flags \ttt{timeFlag} and \ttt{detectFlag} are used.
The \ttt{WaitForPreconditions} state has two internal transitions.
One is triggered by a signal event associated with the signal C and calls a transition effect to update \ttt{detectFlag} to true.
The other one triggered by a time event sets \ttt{timeFlag} to true. 
The expression associated with the change event updates from false to true once two flags \ttt{timeFlag} and \ttt{detectFlag} are set to true. 
The periodic evaluation time is configured as 10ms.
%\ttt{timeFlag} is set by executing the effect \ttt{setTime} of the internal transition of \ttt{WaitForPreconditions} triggered by the \ttt{TE\_MIN} time event. 
%And \ttt{detectFlag} is set by the method \ttt{setDetect} once a \ttt{DetectorOn} event occurs to trigger another internal transition of \ttt{WaitForPreconditions}.
\begin{figure}
	\centering
	\includegraphics[clip, trim=0.0cm 11.5cm 22.1cm 0cm, width=0.7\columnwidth]{figures/highwayopenalternativesoptimized}
	\caption{Alternative state machine designs for the \ttt{HighwayOpen} state} 
	\label{fig:highwayopenalternatives}
\end{figure}

%The architect creates the USM (using a modeling tool) for \ttt{Intersection} 
%since its behavior is complex and needs graphical tools to represent 
%for better understanding.
%For the programmer, he 
%is more familiar with C++. He 
%develops the low-level behavior and creates the state machine for \ttt{TrafficLight} textually, from scratch.
%These actors worked in parallel and synchronized their assignment after finishing.
%The synchronization is realized by our previously presented process.

For simulation of TLC, we reuse the detector class developed in \cite{trafficlight} to automatically generate \ttt{DetectorOn/DetectorOff} signals. 

The support of UML events (change events and time events) and deferred events does not only provide designers more options to specify but also simplify system behaviors. 
It can also reduce the number of states.
For example, the numbers of sub-states of \ti{HighwayOpen} with the use of deferred events and change events are two and one, respectively, while Yasmine requires three states.   
However, deferred events might make the design more difficult to understand because of its specialized semantics.  

\begin{comment}
\begin{table}[]
	\small
	\centering
	\caption{My caption}
	\label{my-label}
	\begin{tabular}{|l|l|l|l|}
		\hline
		\multirow{2}{*}{Criteria} & \multirow{2}{*}{Yasmine} & \multicolumn{2}{c|}{Our tool}                                                 \\ \cline{3-4} 
		&                          & Normal GCC & \begin{tabular}[c]{@{}l@{}}GCC with \\ optimization\end{tabular} \\ \hline
		LoC                       &                          &            &                                                                  \\ \hline
		Binary size               &                          &            &                                                                  \\ \hline
	\end{tabular}
\end{table}
\end{comment}

%show the state machine and associated code

%show some comparison with the implementation of yasmine: line of code, binary size

%\input{sections/lego}
	
	\section{Related work}
\label{sec:relatedwork}
Implementation and code generation techniques for USMs are closely related to the forward engineering of our RTE.
%Main approaches including switch/if, state table and state pattern are investigated.

%Switch/if is the most intuitive technique implementing a "flat" state machine. Two types of switch/if are supported. The first one uses a scalar variable representing the current active state \cite{Booch1998}. A method for each event processes the variable as a discriminator in switch/if statement. The second one uses a double nested switch/if and has two variables to represent the current active state and the event to be processed \cite{Douglass1999}. The latter are used as the discriminators of an outer switch statement to select between states and an inner one/if statement to decide how the event should be processed. The behavior code of the two types is put in one file or class. This practice makes code cumbersome, complex, difficult to read and less explicit when the number of states grows or the state machine is hierarchical. Furthermore, the first approach lets the code scatter in different places. Therefore, maintaining or modifying such code of complex systems is very difficult.

Switch/if is the most intuitive technique implementing a "flat" state machine. The latter can be implemented by either
using a scalar variable \cite{Booch1998} and a method for each event or using two variables as the current active state and the incoming event used as the discriminators of an outer switch statement to select between states and an inner one/if statement, respectively. The double dimensional state table approach \cite{Douglass1999} uses one dimension represents states and the other one all possible events. 
%Each cell of the table is associated with a function pointer meaning that the state associated with a dimension index of the cell is triggered by the event associated with the other dimension. 
The behavior code of these techniques is put in one file or class. This practice makes code cumbersome, complex, difficult to read and less explicit when the number of states grows or the state machine is hierarchical. 
%Therefore, maintaining or modifying such code of complex systems is very difficult. 
Furthermore, these approaches requires every transition must be triggered by at least an event. This is obviously only applied to a small sub-set of USMs.  

State pattern \cite{Shalyto2006,Douglass1999} is an object-oriented way to implement flat state machines. Each state is represented as a class and each event as a method. %The event is processed by a delegation from the context class to sub-states. 
Separation of states in classes makes the code more readable and maintainable. %Unfortunately, this technique only supports flat state machines. 
This pattern is extended in \cite{niaz_mapping_2004} to support hierarchical-concurrent USMs. However, the maintenance of the code generated by this approach is not trivial since it requires many small changes in different places. %This is impractical when dealing with large state machines. %Furthermore, similar to the state table, this approach also poses the requirement of having at least one event for transition.

Many tools, such as \cite{ibm_rhapsody, sparxsystems_enterprise_2014}, apply these approaches to generate code from USMs. Readers of this paper are recommended referring to \cite{Domnguez2012} for a systematic survey on different tools and approaches generating code from USMs.

Double-dispatch (DD) pattern in \cite{spinke_object-oriented_2013} in which %as a new technique to implement state machines. 
represent states and events as classes. Our generation approach relies on and extends this approach. The latter profits the polymorphism of object-oriented languages. %provides some 1-1 mappings from state machines to object-oriented code and the implementation technique 
%is not dependent on a specific programming language. 
However, DD does not deal with triggerless transitions and different event types supported by UML such as \ti{CallEvent}, \ti{TimeEvent} and \ti{SignalEvent}. Furthermore, DD is not a code generation approach but an approach to manually implementing state machines.
	
	\section{\uppercase{Conclusion}}
\label{sec:conclusion}
We presented an approach whose objective is to provide a complete, efficient, and UML-compliant code generation from UML State Machines with full features. 
The design for concurrency of generated code is based on multi-thread of POSIX.
The code generation pattern extends the IF-ELSE/SWITCH patterns and uses a hierarchical structure to preserve the state machine hierarchy. 
%The hierarchy of USM is kept by our simple state structure.

We implemented our pattern as part of the Papyrus modeling tool. 
We evaluated our tool by conducting experiments on the semantic-conformance and efficiency of generated code.
The conformance is tested under PSSM: 62 of 66 tests passed.
These results are a good hint that our tool preserves the UML State Machine semantics during code generation.
For efficiency, we used the benchmark defined by the Boost library to compare code generated by our tool to other approaches.
The results showed that our tool produces efficient code that runs fast in event processing speed and is small in executable size.

Code produced by our tool, however, consumes slightly more memory than that of the others at runtime.
%Furthermore, some PSSM tests are failed.
In future work, we will fix this issue by making multi-thread part of generated code more concise.  
Furthermore, we will use the pattern to support Java code generation from UML State Machines. 
	
	
	%\section*{\uppercase{Acknowledgements}}
	%==removed for blind review==
	
	
	\balance
	\Urlmuskip=0mu plus 1mu\relax
	{
	\bibliographystyle{apalike}
	\bibliography{refs}
}
	
	%\section*{\uppercase{Appendix}}
	
	%\noindent If any, the appendix should appear directly after the
	%references without numbering, and not on a new page. To do so please use the following command:
	%\textit{$\backslash$section*\{APPENDIX\}}
	
	\vfill
\end{document}
