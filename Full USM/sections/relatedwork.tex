\section{Related work}
\label{sec:relatedwork}
Implementation and code generation techniques for USMs are closely related to the forward engineering of our RTE.
%Main approaches including switch/if, state table and state pattern are investigated.

%Switch/if is the most intuitive technique implementing a "flat" state machine. Two types of switch/if are supported. The first one uses a scalar variable representing the current active state \cite{Booch1998}. A method for each event processes the variable as a discriminator in switch/if statement. The second one uses a double nested switch/if and has two variables to represent the current active state and the event to be processed \cite{Douglass1999}. The latter are used as the discriminators of an outer switch statement to select between states and an inner one/if statement to decide how the event should be processed. The behavior code of the two types is put in one file or class. This practice makes code cumbersome, complex, difficult to read and less explicit when the number of states grows or the state machine is hierarchical. Furthermore, the first approach lets the code scatter in different places. Therefore, maintaining or modifying such code of complex systems is very difficult.

Switch/if is the most intuitive technique implementing a "flat" state machine. The latter can be implemented by either
using a scalar variable \cite{Booch1998} and a method for each event or using two variables as the current active state and the incoming event used as the discriminators of an outer switch statement to select between states and an inner one/if statement, respectively. The double dimensional state table approach \cite{Douglass1999} uses one dimension represents states and the other one all possible events. 
%Each cell of the table is associated with a function pointer meaning that the state associated with a dimension index of the cell is triggered by the event associated with the other dimension. 
The behavior code of these techniques is put in one file or class. This practice makes code cumbersome, complex, difficult to read and less explicit when the number of states grows or the state machine is hierarchical. 
%Therefore, maintaining or modifying such code of complex systems is very difficult. 
Furthermore, these approaches requires every transition must be triggered by at least an event. This is obviously only applied to a small sub-set of USMs.  

State pattern \cite{Shalyto2006,Douglass1999} is an object-oriented way to implement flat state machines. Each state is represented as a class and each event as a method. %The event is processed by a delegation from the context class to sub-states. 
Separation of states in classes makes the code more readable and maintainable. %Unfortunately, this technique only supports flat state machines. 
This pattern is extended in \cite{niaz_mapping_2004} to support hierarchical-concurrent USMs. However, the maintenance of the code generated by this approach is not trivial since it requires many small changes in different places. %This is impractical when dealing with large state machines. %Furthermore, similar to the state table, this approach also poses the requirement of having at least one event for transition.

Many tools, such as \cite{ibm_rhapsody, sparxsystems_enterprise_2014}, apply these approaches to generate code from USMs. Readers of this paper are recommended referring to \cite{Domnguez2012} for a systematic survey on different tools and approaches generating code from USMs.

Double-dispatch (DD) pattern in \cite{spinke_object-oriented_2013} in which %as a new technique to implement state machines. 
represent states and events as classes. Our generation approach relies on and extends this approach. The latter profits the polymorphism of object-oriented languages. %provides some 1-1 mappings from state machines to object-oriented code and the implementation technique 
%is not dependent on a specific programming language. 
However, DD does not deal with triggerless transitions and different event types supported by UML such as \ti{CallEvent}, \ti{TimeEvent} and \ti{SignalEvent}. Furthermore, DD is not a code generation approach but an approach to manually implementing state machines.