\section{\uppercase{Features}}
\label{sec:uniqueness}
Our pattern and tool has some features compared to other tools as followings:

\vskip 0.1cm
\noindent
\tb{Completeness:} Our tool supports all state machine vertexes and transitions including all pseudo states and transition kinds such as external, local, and internal. 
Hence, the tool improves flexibility of using UML State Machines to express architecture behavior.
For the moment, our tool cannot deal with transitions from an \ti{entry point} to an \ti{exit point}.
We believe that these transitions are not used in reality.
This is because the contradictory semantics of \ti{entry points} and \ti{exit points}. 
In UML, \ti{entry points} and \ti{exit points} represent entering points and exit points of a compoiste state, respectively. 
They provide encapsulation of the insides of the state. 
The \ti{entry points} allow users to customize the way to enter the composite state instead of the default entering way while the \ti{exit points} allow to customize the exiting way.
For example, the \ti{Enp} entry point in Fig. \ref{fig:entering} allows the \ti{S5} sub-state of the \ti{S1} composite state to be active instead of \ti{S3} by the default entering way.

\vskip 0.1cm
\noindent	
\tb{Event support:} Our tool promotes four UML event types and event deference mechanism, which are able to express synchronous and asynchronous behaviors and exchange data between components/classes.

\vskip 0.1cm
\noindent	
\tb{UML-conformance:} A recent specification formalizing the Precise Semantics of UML State Machine (PSSM) is under standardization of the OMG.
It defines a test suite with 66 test cases for validating the conformance of runtime execution of code generated from UML State Machines.
We have experimented our tool with the test suite.
Traced execution results of 62/66 test cases comply with the standard and are, therefore, a good hint that the execution is semantically correct.
%Due to space limitation, the details of patterns and evaluation for state machine code generation semantics are not presented in this paper.

\vskip 0.1cm
\noindent	
\tb{State machine configuration:} 
Asynchronous events such as signal events, change events, and time events are stored in an event queue.
%Change expressions of change events are monitored and periodically evaluated\footnote{Currently, this discrete evaluation mechanism is not recommended for critical systems since the monitor might miss change event occurrences between two evaluations.} to track their values.
A signal event can bring data (message).
Our tool allows to configure the event queue size and the maximal size of signals.
The configuration is not specified by UML because the specification wants to be abstract.
We allow to determine these values through a specific profile.
Note that the configuration information might not be needed in dynamic memory allocation.
The latter, however, is not recommended in embedded systems.
%Fig. \ref{fig:entering} shows the configuration stereotype annotated on the state machine example.

\vskip 0.1cm
\noindent
\tb{Efficiency:} We conducted experiments on some benchmarks to show that code generated by our tool is efficient and can be used to develop resource-constrained embedded software.
Specifically, event processing is fast and the size of executable files compiled from generated code is small.

\vskip 0.1cm
\noindent
\tb{Event API:}
Generated code in our tool provides APIs for environment code to invoke operations or send data signals to reactive classes.
The invocations and sending will automatically fire events for state machines to process.

\vskip 0.1cm
\noindent
\tb{Concurrency:} 
Concurrency aspects in state machines including doActivity of states, orthogonal regions, event detection, and event queue management are handled by the execution of multiple threads.
Currently, we use POSIX threads for concurrency.

\vskip 0.1cm
\noindent
\tb{Portability:}
Currently, our tool generates C++ code.
The generated code can run on POSIX systems such as Ubuntu without installing any additional libraries to be able to compile and execute the code.
Our code generation pattern and tool can be extended to generate code in other programming languages such as Java which supports threads and mutexes for multi-thread synchronization.