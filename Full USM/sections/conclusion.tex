\section{\uppercase{Conclusion}}
\label{sec:conclusion}
We presented an approach whose objective is to provide a complete, efficient, and UML-compliant code generation from UML State Machines with full features. 
The design for concurrency of generated code is based on multi-thread of POSIX.
The code generation pattern extends the IF-ELSE/SWITCH patterns and uses a hierarchical structure to preserve the state machine hierarchy. 
%The hierarchy of USM is kept by our simple state structure.

We implemented our pattern as part of the Papyrus modeling tool. 
We evaluated our tool by conducting experiments on the semantic-conformance and efficiency of generated code.
The conformance is tested under PSSM: 62 of 66 tests passed.
These results are a good hint that our tool preserves the UML State Machine semantics during code generation.
For efficiency, we used the benchmark defined by the Boost library to compare code generated by our tool to other approaches.
The results showed that our tool produces efficient code that runs fast in event processing speed and is small in executable size.

Code produced by our tool, however, consumes slightly more memory than that of the others at runtime.
%Furthermore, some PSSM tests are failed.
In future work, we will fix this issue by making multi-thread part of generated code more concise.  
Furthermore, we will use the pattern to support Java code generation from UML State Machines. 