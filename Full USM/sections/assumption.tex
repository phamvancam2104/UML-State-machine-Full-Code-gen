\subsection{Assumption}
%todo: give some assumptions on code generation such as functions to create methods, attriutes, classes
Assuming that we want to generate from the state machine to an object oriented programming language $ActLang$, which is a C++-like and supports multi-threading by following functions and resource control as mutexes.
\begin{itemize}
	%\item $genClass(n, generals, itfs)$ creates a class with its name, parent class set, and implemented interfaces as \ti{n}, \ti{generals}, and \ti{itfs}.
	
	%\item $genMtd(n, c, type, params)$ creates a method $m$ with its name as $n$ inside the class $c$, its return type as $type$, and $params$ as its parameter set.
	
	%\item $genAttr(n, c, type, multiplicity)$ creates an attribute named $n$ in the class $c$ and typed by $type$. The create attribute is an array if $multiplicity > 1$, otherwise a simple attribute.
	
	%\item $genEnum(n)$ and $genEnumLit(enum, n)$ create an enumeration and its enumeration literal, respectively.
	
	%\item $genBody(m, body)$ adds a body to a method. The body is a string which contains a list of statements.
	
	%\item $createParalle(t, seg)$ generates a mechanism which allows the segment code $seg$ run in a thread $t$. Similarly, $genWait(t), genJoin(t)$.
	
	\item A mutex has three methods $lock$, $unlock$, and $wait$, which automatically unlocks the mutex and waits until it receives a signal.  
	
	%\item $synchronize(seg)$ generates a mechanism which allows the segment code $seg$ run safely (can be either based on \ti{POSIX pthread} or \ti{Java synchronize} mechanism).
	
	%\item $toString(stts)$ is used to convert a list of statements $stts$ into a readable string which can be add to a method as its body.
	
	%\item Concatenation of two strings $str1$ and $str2$ is concisely described as $str1 + str2$.
	
	%\item \ti{WHILE}, \ti{FOR} \ti{IF}, \ti{ELSE} are symbols representing while and for loops, if and else statements.
	
	\item \ti{FORK(func)} creates a thread (lightweight process) associated with the function/method \ti{func} and \ti{JOIN(theThread)} waits until the method associated with the thread \ti{theThread} completes.
\end{itemize} 
