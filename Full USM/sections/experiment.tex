\section{experiments}
\label{sec:exp}
\subsection{Semantic conformance of runtime execution}
\label{subsec:exp1}
%\paragraph{Bisimulation for semantic-conformance}
To evaluate the semantic conformance of runtime execution of generated code, we use a set of examples provided by Moka \cite{moka}. Moka is a model execution engine offering Precise Semantics of UML Composite Structures \cite{OMG2015}. Fig. \ref{fig:semanticconformance} shows our method. We first use our code generator to generate code (Step (1)) from the Moka example set. Step (2) simulates the examples by using Moka to extract the sequence (\ti{SimTraces}) of observed traces including executed actions. The sequence (\ti{RTTraces}) of traces is also obtained by the runtime execution of the code generated from the same state machine in a Step (3). The generated code is semantic-conformant if the sequences of traces are the same for both of the state machine and generated code \cite{Blech2005}. The current version of Moka does not support simulation for \ti{TimeEvent} and history pseudo states, we therefore leave experiments for \ti{TimeEvent} as future work.



Within our scope as previously defined 30 examples of the Moka example set are tested. \ti{SimTraces} and \ti{RTTraces} for each case are the same. 
This indicates that, within our study scope, the runtime execution of code generated by our generator can produce traces semantically equivalent to those obtained via simulation. 

\begin{figure}
	\centering
	\includegraphics[clip, trim=0.2cm 8.6cm 19.4cm 6.9cm, width=\columnwidth]{figures/semanticconformance.pdf}
	\caption{Semantic conformance evaluation methodology} 
	\label{fig:semanticconformance}
\end{figure}		

