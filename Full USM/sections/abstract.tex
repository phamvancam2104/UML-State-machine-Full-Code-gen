\begin{abstract}
Event-driven architecture is an useful way to design and solve the complexity of today systems.
Unified Modeling Language State Machine and its visualization are a powerful means to the modeling of the logical behavior of such architecture.   
Over the years of research and development, many approaches focus on reducing the gap between the modeling and implementation world. 
However, despite many useful UML State Machine concepts for modeling and designing complex system, the support of existing generation approaches is still limited to simple cases, especially when considering concurrency of \ti{doActivity} and orthogonal regions, pseudo states such as history, and different event types.

In order for wider integrating MDE into software development today, one task, among other important tasks, is needed to generate code for all modeling concepts with respect to the semantics. 
This paper presents a code generation approach whose objective is to clean the above issues.
The approach combines IF-ELSE constructions of programming languages and the state pattern with our support for concurrency. 
Even supporting multi-thread-based concurrency, binary files compiled from and the event processing speed of runtime execution of code generated by our approach are smaller 20 and 30 [?] times, respectively, in comparison with to the boost library ==> should have compare with others.
\end{abstract}