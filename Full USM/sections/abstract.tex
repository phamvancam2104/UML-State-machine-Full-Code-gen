\begin{abstract}
The so-called model-driven engineering approach relies on two paradigms, abstraction and automation, recognized as very efficient for dealing with complexity of today system. 
%Abstraction is the ability to provide simplified and focused view of a system and requires adequate modeling language. 
%For this concern, it is clear that the Unified Modeling Language (UML) is nowadays the most used, educated, documented and tooled modeling language. 
%Even, if a graphical language such as the UML is not the silver bullet for all software related concerns, it provides hence better support than text-based solutions for some concerns such as architecture and logical behavior of application development. 
UML state machine and their visual representations are much more suitable to describe logical behaviors of system entities than any equivalent text based description such as IF-THEN-ELSE or SWITH-CASE constructions. Although many industrial tools and research prototypes can generate executable code from such graphical language, their support only focus special cases of UML state machine, especially non-concurrent state machines. While UML state machine concepts such as concurrency, pseudo states such as history, fork, junction and time events are widely used by software architects, the code generation of the existing approaches does not take these concepts into account. To blur the boundaries between the modeling and coding worlds, a code generator should be able to generate code from all concepts of modeling languages with respect to the semantics described by the language specifications.

To this end, this paper proposes an approach to generate code from all UML state machines with all of its concepts with respect to the Precise Semantics of UML, especially concurrent state machines which are based on multi-thread-based design.
%After code modifications, round-trip engineering is needed to make the model and code consistent, which is a critical aspect to meet quality and performance constraint required from project manager today. Unfortunately, current UML tools only support structural concepts for round-trip engineering such as those available from class diagrams. In this paper, we address the round-trip engineering of UML state-machine and its related generated code. We propose a round-trip engineering approach consisting of a forward process which generates code by using transformation patterns, and a backward process which is based on code pattern detection to update the original state machine model from the modified generated code. We implemented a prototype and conducted several experiments on different aspects of the round-trip engineering to verify the proposed approach.
%\lipsum[1]
\end{abstract}